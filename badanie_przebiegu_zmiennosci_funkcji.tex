\documentclass[a4paper,10pt]{article}
\usepackage[utf8]{inputenc}
\usepackage{polski}
\usepackage{amsmath,amsfonts,amsthm}
\usepackage{amssymb,latexsym}
\usepackage{graphicx}
\usepackage{geometry}
\usepackage{subcaption}
\usepackage{listings}
\usepackage[autostyle]{csquotes}
\usepackage{xcolor}
 \MakeAutoQuote{‘}{’}
\usepackage{algorithmicx}
\usepackage{fancyvrb}

\title{Badanie przebiegu zmienności funkcji\\ \large{Projekt w wxMaxima}}


\newgeometry{tmargin=2.5cm, bmargin=3cm, lmargin=2cm, rmargin=2cm}
\begin{document}
\maketitle
\begin{center}
\includegraphics{"logo.jpg"}\\
\vspace{200pt}
\author{Emil Szewczak}\\
{\sl Kierunek: Inżynieria i Analiza Danych}
\end{center}

\newpage
\tableofcontents

\newpage
	
\section{Wstęp}
	Tematem projektu jest badanie przebiegu zmienności funkcji.\\
Należy napisać program w MAXIMIE, dzięki któremu zostanie zbadany przebieg wybranej funkcji.\\ Kryteria oceny:
\begin{itemize}
	\item Warunek konieczny: funkcja musi posiadać co najmniej jedno ekstremum i co najmniej jednąasymptotę, a program ma działać.
	\item Własności funkcji ponad warunek konieczny.
	\item Wystąpienie wszystkich etapów badania przebiegu (bez własności specjalnych - parzystość, nie-parzystość, okresowość)
	\item Właściwe skomentowanie każdego etapu.
	\item Wykorzystanie możliwości Maximy w celu automatyzacji obliczeń.
	\item Rysunek i jego elementy.
\end{itemize}
Poprzez analizę przebiegu zmienności funkcji rozumiem:  
\begin{enumerate}
\item Wyznaczeniu dziedziny funkcji.
\item Wyznaczeniu miejsc zerowych funkcji (jeśli istnieją).
\item Wyznaczeniu asymptot funkcji (asymptoty pionowe, poziome lub ukośne).
\item Obliczeniu pochodnej funkcji i wyznaczeniu przedziałów monotoniczności funkcji (gdzie funkcja jest rosnąca lub malejąca itp.).
\item Wyznaczeniu ekstremów lokalnych funkcji korzystając z pochodnej funkcji.
\item Obliczeniu drugiej pochodnej funkcji i wyznaczeniu przedziałów wypukłości i wklęsłości funkcji oraz punkty przegięcia.
\item Narysowania szkic wykresu funkcji. Wykres nie musi być dokładny, ale powinien uwzględniać charakterystyczne cechy funkcji opisane w poprzednich punktach (np. monotoniczność, asymptoty itd.).
\item Utworzenie tabelki zmienności dla funkcji.
\end{enumerate}

\section{Funkcja}
Wzór badanej funkcji: \\
\includegraphics{fkod.png}

\section{Dziedzina}
Aby policzyć dziedzinę funkcji $ f(x)=\frac{x^3}{1-x^2} $:
Musimy przyrównać miaownik do zera. Najpierw do zmiennej przypisuje wszystko pod kreską ułamkową:\\
\includegraphics{mianownik.png}\\
Następnie przyrównuje mianownik do zera, aby policzyć jakich argumentów funkcja nie może przyjąć:\\
\includegraphics{dziedzina.png}\\
Zatem funkcja nie ma wartości w punktach 1 i -1. \\


Ostatecznie dziedziną funkcji $f$ jest zbiór:
 $ D: x \in \mathbb{R} \setminus \{ -1,1 \} $.

\section{Miejsca zerowe}
Aby policzyć miejsca zerowe funkcji $ f(x)=\frac{x^3}{1-x^2} $ należy przyrównać funkcję do 0:\\
\includegraphics{zerowe.png}\\
Punkt $x=0$ jest punktem przecięcia się wykresu funkcji z osią Ox.\\


Miejsce zerowe funkcji $f$ to: $x=0$


\section{Asymptoty}

\subsection{Asymptoty pionowe}
Asymptoty pionowe - proste pionowe, przechodzące przez punkty nieciągłości funkcji (granice funkcji w tych punktach są rozbieżne).\\
Z dzidziny wynika, że funkcja jest nieciągła w punktach $x=-1 \land x=1$. Zbadajmy granice lewostronne i prawostronne w tych punktach.\\

Dla $x=1$:

\begin{itemize}
	\item Granica lewostronna:\\
	\includegraphics{lewa1.png}
	\item Granica prawostronna:\\
	\includegraphics{prawa1.png}
\end{itemize}
Granice są rozbieżne więc w punkcje $x=1$ jest asymptota pionowa.\\

Dla $x=-1$:

\begin{itemize}
	\item Granica lewostronna:\\
	\includegraphics{lewa2.png}
	\item Granica prawostronna:\\
	\includegraphics{prawa2.png}
\end{itemize}
Granice są rozbieżne więc w punkcje $x=-1$ jest asymptota pionowa.\\

Ostatecznie funkcja $f$ ma dwie asymptoty pionowe: $x=-1 \land x=1$.

\subsection{Asymptoty poziome}
Asymptoty poziome istnieją, jeżeli granice w $-\infty $ oraz $\infty $ są skończone.\\

Zbadajmy granicę funkcji w $\infty $:\\
\includegraphics{pozioma1.png}

oraz w $-\infty $:\\
\includegraphics{pozioma2.png}\\

Obie te granice nie są skończone dlatego funkcja $f$ nie ma asymptot poziomych.

\subsection{Asymptoty ukośne}
Asymptota ukośna to prosta postaci $ax+b$.Istieją asymptoty lewostronne i prawostronne. Aby one istniały granice z $f-(ax+b)$ przy $-\infty$ lub $\infty$ muszą być równe zero.\\

Zbadajmy asymptotę ukośną prawostronną.\\ \\
Znajdzmy najpierw współczynnik $a$ prostej, badając taką granice:\\
\includegraphics{ukos1.png}\\
Zbadajmy teraz granicę z funkcji minus $ax + 0$:\\
\includegraphics{ukos2.png}\\

Ta granica jest równa 0, dlatego $-x$ jest asymptotą ukośną prawostronną.\\

Zbadajmy asymptotę ukośną lewostronną.\\ \\
Znajdzmy najpierw współczynnik $a$ prostej, badając taką granice:\\
\includegraphics{ukos3.png}\\
Zbadajmy teraz granicę z funkcji minus $ax + 0$:\\
\includegraphics{ukos4.png}\\

Ta granica jest równa 0, dlatego $-x$ jest asymptotą ukośną lewostronną.\\

Z tego wynika, że prosta $-x$ jest asymptotą ukośną obustronną funkcji $f$.

\section{Pierwsza pochodna}
Pochodna funkcji $ f(x)=\frac{x^2-1}{x^2-2} $:\\
\includegraphics{pochodna1.png}\\
Po uproszczeniu:\\
\includegraphics{pochodna2.png}\\

Zatem ostatecznie $ f'(x)=-\frac{x^4-3x^2}{x^4-2x^2+1} $.\\

\subsection{Wyznaczenie przedziału monotoniczności}
Aby wyznaczyć przedziały monotoniczności funkcji sprawdzam kiedy pierwsza pochodna jest mniejsza lub większa od zera. Funkcja jest rosnąca gdy jej pochodnia jest dodatnia, a malejąca gdy ujemna.\\

Funkcja $f$ jest rosnąca dla:\\
\includegraphics{przedzial1.png}\\

a  malejąca dla:\\
\includegraphics{przedzial2.png}\\

Co łatwiej jest odczytać z wykresu pochodnej:\\
\includegraphics[height=13cm]{wykpoch1.png}

Oznacza to, że funkcja $f$ dla $ x \in (-\sqrt{3},-1)\cup(-1,0)\cup(0,1)\cup(1,\sqrt{3})$ rośnie, a dla $ x \in (-\infty,-\sqrt{3})\cup(\sqrt{3},\infty)$ maleje.

\section{Ekstrema lokalne funkcji}
Funkcja może mieć ekstremum tylko w tych miejscach gdzie jej pochodna się zeruje. Dodatkowo aby ekstremum istniało, to funkcja musi w danym punkcie zmienić monotoniczność.\\

W przypadku funkcji $f$:\\
\includegraphics{pochodna3.png}\\


Zatem punkty $ x=-\sqrt{3} \land x=\sqrt{3} \land x=0 $ są podejrzane o ekstremum. Funkcja $f$ zmienia swoją monotoniczność tylko w punktach $ x=-\sqrt{3} \land x=\sqrt{3} $.\\

Funkcja $f$:
\begin{itemize}
	\item $f < 0$, dla $x<-\sqrt{3}$,
	\item $f > 0$, dla $x>-\sqrt{3}$,
\end{itemize}
Zatem w punkcje $x=-\sqrt{3}$ jest minimum lokalne.\\

Wartość ekstremum to:\\
\includegraphics{min.png}\\

Funkcja $f$:
\begin{itemize}
	\item $f > 0$, dla $x<\sqrt{3}$,
	\item $f < 0$, dla $x>\sqrt{3}$,
\end{itemize}
Zatem w punkcje $x=-\sqrt{3}$ jest maksimum lokalne.\\

Wartość ekstremum to:\\
\includegraphics{max.png}


\section{Druga pochodna}
Druga pochodna funkcji  $ f $:\\
\includegraphics{2pochodna1.png}\\
Po uproszczeniu:\\
\includegraphics{2pochodna2.png}\\

\subsection{Przedziąły wypukłości i wklęsłości}
Funkcja $f(x)$ jest wypukła, jeśli $f''(x)>0 $:\\
\includegraphics{wypuklosc.png}\\
\newpage
Funkcja $f(x)$ jest wklęsła, jeśli $f''(x)<0 $:\\
\includegraphics{wkleslosc.png}\\

Co łatwiej odczytać z wykresu drugiej pochodnej:\\
\includegraphics[height=13cm]{wykpoch2.png}

Funkcja $f$ jest wypukła dla $x \in (-\infty, -1)\cup(0,1)$, a wklęsła dla $x \in (-1,0)\cup(1,\infty)$.

\subsection{Punkty przegięcia}
Punkty przegięcia występują w tych miejscach, w których funkcja zmienia wypukłość, czyli $f''(x)=0$:\\
\includegraphics{pochodna4.png}\\

Zatem funkcja $f$ ma jeden rzeczywisty punkt przegięcia $x=0$


\section{Tabela zmienności funkcji}
\includegraphics[width=17cm]{tabela.png}


\section{Wykres funkcji}
\includegraphics[height=13cm]{wykres.png}
Narysowana w MAXIMIE za pomocą tego kodu:\\
\includegraphics{kod.png}

\end{document}

